\documentclass[12pt]{article}
\usepackage[utf8]{inputenc}
\usepackage[T1]{fontenc}
\usepackage{amsmath}
\usepackage{amssymb}
\usepackage{graphicx}
\usepackage{subcaption}
\usepackage[hidelinks]{hyperref}
\usepackage{geometry}
\usepackage{siunitx}
\usepackage{listings}
\usepackage{xcolor}
\usepackage{float}
\usepackage{booktabs}
\usepackage{microtype}
\usepackage[font=small,labelfont=bf]{caption}
\usepackage{cleveref}

\geometry{margin=1in}
\sisetup{inter-unit-product = \ensuremath{{}\cdot{}}}

\title{\textbf{MECH 579 Final Project}\\ \large 2D Heat Equation Optimization via Finite Differences and Automatic Differentiation}
\author{Aidan Kimberley \and Lucas Bessai \\ \small McGill University}
\date{\today}

\begin{document}

\maketitle

\section{Introduction}

This report details the numerical optimization of a 2D heat conduction problem to minimize a weighted objective function of maximum temperature and fan power consumption. We compare two gradient computation techniques: Finite Differences (FD) and Automatic Differentiation (AD) using JAX. The optimization is constrained by a global heat generation requirement of \SI{10}{W} on a $0.04 \times \SI{0.04}{m}$ silicon chip.

\section{Mathematical Formulation}

\subsection{Governing Equations & Optimization}
The system is governed by the 2D Poisson equation with a spatially variable source term $q(x,y) = ax + by + c$. The optimization problem is defined as:

\begin{equation}
    \min_{\mathbf{u}} J(\mathbf{u}) = \omega_1 \frac{\max(T)}{\SI{273.15}{K}} - \omega_2 \eta(v) \quad \text{s.t.} \quad \int_{\Omega} q \, dV = \SI{10}{W}
\end{equation}

where $\mathbf{u} = [v, a, b, c]^\intercal$ contains the fan velocity and heat generation coefficients, and $\eta(v)$ is the fan efficiency curve.

\subsection{Gradient Computation}
Two methods are employed to compute $\nabla J$:
\begin{enumerate}
    \item \textbf{Finite Differences (FD):} Uses forward differences with step size $h$.
    \item \textbf{Automatic Differentiation (AD):} Uses \texttt{jax.grad} to compute exact gradients via the computational graph.
\end{enumerate}

\section{Numerical Analysis and Validation}

\subsection{Gradient Convergence}
To validate the FD approach, we compared it against the exact AD gradients. \Cref{fig:fd_convergence} illustrates that a step size of $h=10^{-3}$ provides the optimal balance between truncation error and floating-point cancellation.

\begin{figure}[H]
    \centering
    \includegraphics[width=0.75\textwidth]{plots/fd_convergence.png}
    \caption{Convergence of Finite Difference gradients to the exact AD baseline.}
    \label{fig:fd_convergence}
\end{figure}

\section{Optimization Results (Finite Differences)}

The optimization was performed using the SciPy \texttt{trust-constr} algorithm.

\subsection{Convergence History}
The solver successfully reached KKT convergence. The objective function decreases monotonically (\cref{fig:objective_convergence}), and the Lagrangian gradient vanishes (\cref{fig:grad_lagrangian}), indicating a local minimum.

\begin{figure}[H]
    \centering
    \begin{subfigure}{0.48\textwidth}
        \includegraphics[width=\textwidth]{plots/objective_function.png}
        \caption{Objective function minimization}
        \label{fig:objective_convergence}
    \end{subfigure}
    \hfill
    \begin{subfigure}{0.48\textwidth}
        \includegraphics[width=\textwidth]{plots/gradient_lagrangian.png}
        \caption{Gradient of the Lagrangian}
        \label{fig:grad_lagrangian}
    \end{subfigure}
    \caption{Optimization convergence history (FD).}
\end{figure}

\begin{figure}[H]
    \centering
    \includegraphics[width=0.65\textwidth]{plots/gradient_objective.png}
    \caption{Evolution of the Objective Function Gradient (FD).}
    \label{fig:grad_objective}
\end{figure}

\subsection{Design Variable Evolution}
The optimal fan velocity converges to the upper bound of the efficient range.

\begin{figure}[H]
    \centering
    \includegraphics[width=0.7\textwidth]{plots/velocity_path.png}
    \caption{Evolution of fan velocity $v$, converging to $\approx \SI{20}{m.s^{-1}}$.}
    \label{fig:velocity_path}
\end{figure}

The heat generation parameters $a, b, c$ evolve to distribute heat generation spatially. The negative values for $a$ and $b$ shift the heat load toward the origin, leveraging the thermal boundary layer development.

\begin{figure}[H]
    \centering
    \includegraphics[width=0.7\textwidth]{plots/abc_parameters.png}
    \caption{Evolution of heat generation coefficients $a, b, c$.}
    \label{fig:abc_parameters}
\end{figure}

\subsection{Objective Components and Constraints}
The optimization trades off a slight increase in maximum temperature to maximize fan efficiency.

\begin{figure}[H]
    \centering
    \begin{subfigure}{0.48\textwidth}
        \includegraphics[width=\textwidth]{plots/max_temperature.png}
        \caption{Maximum Temperature}
    \end{subfigure}
    \hfill
    \begin{subfigure}{0.48\textwidth}
        \includegraphics[width=\textwidth]{plots/fan_efficiency.png}
        \caption{Fan Efficiency}
    \end{subfigure}
    \caption{Evolution of objective components (FD).}
\end{figure}

The total power constraint was satisfied throughout the process:

\begin{figure}[H]
    \centering
    \includegraphics[width=0.65\textwidth]{plots/power_constraint.png}
    \caption{Satisfaction of the \SI{10}{W} total generation constraint.}
    \label{fig:constraint}
\end{figure}

\subsection{Optimal State}
The final temperature distribution shows the combined effect of the optimized source term and convective cooling.

\begin{figure}[H]
    \centering
    \includegraphics[width=0.75\textwidth]{plots/optimal_temperature_distribution.png}
    \caption{Steady-state temperature field for the optimal design (FD).}
    \label{fig:optimal_temp}
\end{figure}

\section{Validation via Automatic Differentiation (AD)}

To confirm the results, the optimization was repeated using JAX. The AD results are virtually identical to the FD results, validating the correctness of the finite difference implementation.

\subsection{AD Convergence Metrics}
\begin{figure}[H]
    \centering
    \begin{subfigure}{0.48\textwidth}
        \includegraphics[width=\textwidth]{plots/objective_function_AD.png}
        \caption{Objective Function}
    \end{subfigure}
    \hfill
    \begin{subfigure}{0.48\textwidth}
        \includegraphics[width=\textwidth]{plots/gradient_lagrangian_AD.png}
        \caption{Lagrangian Gradient}
    \end{subfigure}
    \par\bigskip
    \begin{subfigure}{0.6\textwidth}
        \centering
        \includegraphics[width=\textwidth]{plots/gradient_objective_AD.png}
        \caption{Objective Gradient}
    \end{subfigure}
    \caption{AD Optimization History. The trajectory matches the FD implementation.}
    \label{fig:ad_history}
\end{figure}

\subsection{AD Design Variables}
\begin{figure}[H]
    \centering
    \begin{subfigure}{0.48\textwidth}
        \includegraphics[width=\textwidth]{plots/design_vars_AD.png}
        \caption{Design Variables}
    \end{subfigure}
    \hfill
    \begin{subfigure}{0.48\textwidth}
        \includegraphics[width=\textwidth]{plots/max_temperature_AD.png}
        \caption{Max Temperature}
    \end{subfigure}
    \par\bigskip
    \begin{subfigure}{0.48\textwidth}
        \centering
        \includegraphics[width=\textwidth]{plots/fan_efficiency_AD.png}
        \caption{Fan Efficiency}
    \end{subfigure}
    \caption{Evolution of design parameters using exact AD gradients.}
    \label{fig:ad_vars}
\end{figure}

\subsection{AD Optimal State}
\begin{figure}[H]
    \centering
    \begin{subfigure}{0.48\textwidth}
        \includegraphics[width=\textwidth]{plots/power_constraint_AD.png}
        \caption{Constraint Satisfaction}
    \end{subfigure}
    \hfill
    \begin{subfigure}{0.48\textwidth}
        \includegraphics[width=\textwidth]{plots/optimal_temperature_AD.png}
        \caption{Optimal Temperature Map}
    \end{subfigure}
    \caption{Final state validation using the JAX solver.}
    \label{fig:ad_state}
\end{figure}

\section{Physical Analysis}

\subsection{Fan Speed Plateau}
A parametric sweep reveals why the optimizer favors $v \approx \SI{20}{m.s^{-1}}$. As seen in \cref{fig:fan_speed_cooling}, cooling performance plateaus beyond \SI{18.6}{m.s^{-1}}. Increasing velocity further yields diminishing thermal returns while degrading fan efficiency.

\begin{figure}[H]
    \centering
    \includegraphics[width=0.85\textwidth]{plots/temp_vs_fanspeed.png}
    \caption{Steady-state temperature response to fan velocity, showing the cooling plateau.}
    \label{fig:fan_speed_cooling}
\end{figure}

\subsection{Pareto Frontier}
The Pareto front (\cref{fig:pareto}) illustrates the trade-off between thermal performance and efficiency. The optimal velocity is remarkably stable at \SI{20}{m.s^{-1}} for the majority of weights, shifting only when temperature is prioritized almost exclusively.

\begin{figure}[H]
    \centering
    \includegraphics[width=0.8\textwidth]{plots/pareto.png}
    \caption{Pareto front showing the trade-off between max temperature and fan efficiency.}
    \label{fig:pareto}
\end{figure}

The numerical results for the three specific weighting scenarios tested are summarized in \Cref{tab:pareto_results}.

\begin{table}[H]
\centering
\caption{Pareto front optimization results for different weight combinations.}
\label{tab:pareto_results}
\begin{tabular}{ccccc}
\toprule
$\omega_1$ & $\omega_2$ & Max Temp. [\unit{\degreeCelsius}] & Efficiency $\eta$ & Velocity $v$ [\unit{m.s^{-1}}] \\
\midrule
0.001 & 0.999 & 48.45 & 0.8000 & 20.00 \\
0.500 & 0.500 & 48.45 & 0.8000 & 20.00 \\
0.999 & 0.001 & 48.44 & 0.7833 & 22.89 \\
\bottomrule
\end{tabular}
\end{table}

\section{Discussion and Conclusions}

The study successfully implemented PDE-constrained optimization using both FD and AD. 
\begin{itemize}
    \item \textbf{Methodology:} AD proved superior for gradient accuracy and ease of use (no step-size tuning), though the JAX implementation required fixed loop iterations (\texttt{jax.lax.scan}) rather than adaptive convergence.
    \item \textbf{Physical Results:} The optimal design leverages the peak of the fan efficiency curve ($\approx \SI{20}{m.s^{-1}}$), as higher velocities provide negligible cooling benefits due to the Nusselt number plateau.
\end{itemize}

\section*{Code Structure}
The implementation is split into \texttt{heat\_eq\_2D\_opt.py} (FD/SciPy) and \texttt{jax\_ad\_heat.py} (AD/JAX).

\end{document}